% $Id: title-LHCb-INT.tex  $
% ===============================================================================
% Purpose: LHCb-INT Note title page template
% Author: P. Koppenburg
% Created on: 2015-05-18
% ===============================================================================

%%%%%%%%%%%%%%%%%%%%%%%%%
%%%%%  TITLE PAGE  %%%%%%
%%%%%%%%%%%%%%%%%%%%%%%%%
\begin{titlepage}

% Header ---------------------------------------------------
\vspace*{-1.5cm}

\noindent
\begin{tabular*}{\linewidth}{lc@{\extracolsep{\fill}}r@{\extracolsep{0pt}}}
\ifthenelse{\boolean{pdflatex}}% Logo format choice
{\vspace*{-2.7cm}\mbox{\!\!\!\includegraphics[width=.14\textwidth]{figs/lhcb-logo.pdf}} & &}%
{\vspace*{-1.2cm}\mbox{\!\!\!\includegraphics[width=.12\textwidth]{lhcb-logo.eps}} & &}
 \\
 & & CERN-LHCb-INT-2016-031 \\  % ID 
 & & \today \\ % Date - Can also hardwire e.g.: 23 March 2010
 & & \\
\hline
\end{tabular*}

\vspace*{4.0cm}

% Title --------------------------------------------------
{\normalfont\bfseries\boldmath\huge
\begin{center}
  Sensitivity of LHCb and its upgrade in the measurement of  \BRof\Kspizmm
\end{center}
}

\vspace*{2.0cm}

% Authors -------------------------------------------------
\begin{center}
V.~Chobanova$^1$, X.~Cid Vidal$^1$, J. P. ~Dalseno$^2$,  M.~Lucio Mart\'inez$^1$,  D.~Mart\'inez Santos$^1$, V.~Renaudin$^3$
\bigskip\\
{\normalfont\itshape\footnotesize
$ ^1$Universidade de Santiago de Compostela, Santiago de Compostela, Spain\\
$^2$University of Bristol, Bristol, The United Kingdom \\
$^3$Laboratoire de l'Accelerateur Lineaire (LAL),  Orsay, France \\
}
\end{center}

\vspace{\fill}

% Abstract -----------------------------------------------
\begin{abstract}
  \noindent
 The sensitivity of the LHCb experiment to \BRof\Kspizmm is analyzed in light of the 2011, 2012 and 2016 data and the 
oportunities the full software trigger of the LHCb upgrade provides. Two strategies are considered: the full reconstruction of the
decay products and the partial reconstruction using only the dilepton pair and kinematic constraints. In both
cases, the sensitivity achieved can surpass the world's current best. Both approaches could be statistically combined, further improving the result.
\end{abstract}

\vspace*{2.0cm}
\vspace{\fill}

\end{titlepage}


\pagestyle{empty}  % no page number for the title 

%%%%%%%%%%%%%%%%%%%%%%%%%%%%%%%%
%%%%%  EOD OF TITLE PAGE  %%%%%%
%%%%%%%%%%%%%%%%%%%%%%%%%%%%%%%%

%  empty page follows the title page ----
\newpage
\setcounter{page}{2}
\mbox{~}

\cleardoublepage
