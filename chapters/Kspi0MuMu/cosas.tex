\documentclass[12pt,a4paper]{article}
%%\documentclass[12pt,letter]{article}
% For two column text, add "twocolumn" as an option to the document
% class. Also uncomment the two "onecolumn" and "twocolumn" lines
% around the title page below.

% Variables that controls behaviour  
\usepackage{ifthen} % for conditional statements
\newboolean{pdflatex}
\setboolean{pdflatex}{true} % False for eps figures 

\newboolean{articletitles}
\setboolean{articletitles}{true} % False removes titles in references

\newboolean{uprightparticles}
\setboolean{uprightparticles}{false} %True for upright particle symbols

\newboolean{inbibliography}
\setboolean{inbibliography}{false} %True once you enter the bibliography

\input{preamble}
\usepackage{longtable} %

\begin{document}
\begin{figure} [htb!]
\begin{center}
\includegraphics[scale=0.20]{figs/DOCAPARTIALptcut.pdf}
\includegraphics[scale=0.20]{figs/mu1ipsPARTIALptcut.pdf}
\includegraphics[scale=0.20]{figs/mu2ipsPARTIALptcut.pdf}
\includegraphics[scale=0.20]{figs/BipsPARTIALptcut.pdf}
\includegraphics[scale=0.20]{figs/BptPARTIALptcut.pdf}
\includegraphics[scale=0.20]{figs/BdissigPARTIALptcut.pdf}
\includegraphics[scale=0.20]{figs/mu2_probNNmuPARTIALptcut.pdf}
\includegraphics[scale=0.20]{figs/mu1_probNNmuPARTIALptcut.pdf}
\includegraphics[scale=0.20]{figs/Vchi2PARTIALptcut.pdf}
\includegraphics[scale=0.20]{figs/mu1_track_Chi2DoFPARTIALptcut.pdf}
\includegraphics[scale=0.20]{figs/mu2_track_Chi2DoFPARTIALptcut.pdf}
\includegraphics[scale=0.20]{figs/mu1_hitsInOTPARTIALptcut.pdf}
\includegraphics[scale=0.20]{figs/mu2_hitsInOTPARTIALptcut.pdf}
\includegraphics[scale=0.20]{figs/mu2_hitsInITPARTIALptcut.pdf}
\includegraphics[scale=0.20]{figs/mu1_hitsInITPARTIALptcut.pdf}
\includegraphics[scale=0.20]{figs/mu1_hitsInTTPARTIALptcut.pdf}
\includegraphics[scale=0.20]{figs/mu2_hitsInTTPARTIALptcut.pdf}
\includegraphics[scale=0.20]{figs/mu1_hitsInVPARTIALptcut.pdf}
\includegraphics[scale=0.20]{figs/mu2_hitsInVPARTIALptcut.pdf}
\includegraphics[scale=0.20]{figs/SV1PARTIALptcut.pdf}
\includegraphics[scale=0.20]{figs/SV2PARTIALptcut.pdf}
\includegraphics[scale=0.20]{figs/SV3PARTIALptcut.pdf}
\includegraphics[scale=0.20]{figs/ProbNNghost1PARTIALptcut.pdf}
\includegraphics[scale=0.20]{figs/ProbNNghost2PARTIALptcut.pdf}
\caption{Input variable distributions for the PARTIAL case, using \Kspipi data with a cut on the $p_t$ ($p_t > 250 \rm GeV$).
\label{fig:MVAhistos_PARTIAL2pipi1}}
\end{center}
\end{figure}


\begin{figure} [htb!]
\begin{center}
\includegraphics[scale=0.20]{figs/mu1GPPARTIALptcut.pdf}
\includegraphics[scale=0.20]{figs/mu2GPPARTIALptcut.pdf}
\includegraphics[scale=0.20]{figs/PIDmu1PARTIALptcut.pdf}
\includegraphics[scale=0.20]{figs/PIDmu2PARTIALptcut.pdf}
\caption{Input variable distributions for the PARTIAL case, using \Kspipi data with a cut on the $p_t$ ($p_t > 250 \rm GeV$).
\label{fig:MVAhistos_PARTIAL2pipi2}}
\end{center}
\end{figure}
\end{document}